\documentclass[journal]{IEEEtran}
%
% If IEEEtran.cls has not been installed into the LaTeX system files,
% manually specify the path to it like:
% \documentclass[journal]{../sty/IEEEtran}

\makeatletter
\def\markboth#1#2{\def\leftmark{\@IEEEcompsoconly{\sffamily}\MakeUppercase{\protect#1}}%
\def\rightmark{\@IEEEcompsoconly{\sffamily}\MakeUppercase{\protect#2}}}
\makeatother

\usepackage[latin1]{inputenc}
\usepackage[T1]{fontenc}
\usepackage{cite}
\usepackage{graphicx}
\usepackage{url}
\usepackage[norsk,english]{babel}
\usepackage{amsmath}
\usepackage{verbatim}
\usepackage{varioref}
\usepackage{color}
\usepackage{listings}

%\usepackage[cmex10]{amsmath}
%\usepackage{algorithmic}
%\usepackage{array}
%\usepackage{mdwmath}
%\usepackage{mdwtab}
%\usepackage[tight,footnotesize]{subfigure}
%\usepackage[caption=false]{caption}
%\usepackage[font=footnotesize]{subfig}
%\usepackage[caption=false,font=footnotesize]{subfig}
%\usepackage{fixltx2e}
%\usepackage{stfloats}

% correct bad hyphenation here
\hyphenation{net-works}


\begin{document}

\title{Implementation of CacheCast\\ in the ns-3 network simulator}

\author{Rizwan~Ali~Ahmed,~
        Kanat~Sarsekeyev,~
        Bekzahan~Kassymbekov,~
        and~Dag~Henning~Liodden~S�rb�}% <-this % stops a space

% \thanks{J. Doe and J. Doe are with Anonymous University.}% <-this % stops a space
% \thanks{Manuscript received April 19, 2005; revised January 11, 2007.}}

% The paper headers
% \markboth{Journal of \LaTeX\ Class Files,~Vol.~6, No.~1, January~2007}%
% {Shell \MakeLowercase{\textit{et al.}}: Bare Demo of IEEEtran.cls for Journals}


%%---------------------------------------------------------------------------%%
% make the title area
\maketitle


\begin{abstract}
The abstract goes here.
\end{abstract}


% \begin{IEEEkeywords}
% IEEEtran, journal, \LaTeX, paper, template.
% \end{IEEEkeywords}


\section{Introduction}
\IEEEPARstart{T}{he} goal of this project...

Bla bla... 



\section{CacheCast}


\section{ns-3}


\section{Common data structures}
In this section we explain the common data structures used by the different 
parts of the ns-3 implementation of CacheCast. These data structures are the 
CacheCast header and the CacheCast packet tag.

\subsection{CacheCast header}
The CacheCast header is represented in ns-3 as a class named CacheCastHeader. 
This class is derived from the general ns3::Header class. The class 

\begin{footnotesize}
\begin{lstlisting}[language=C]
--------------------------------------------------
#include "ns3/header.h"

namespace ns3 {
class CacheCastHeader : public Header
{
public:
  CacheCastHeader ();
  CacheCastHeader (uint32_t payloadId, 
    uint32_t payloadSize, uint32_t index);
  uint32_t GetPayloadId (void) const;
  uint32_t GetPayloadSize (void) const;
  uint32_t GetIndex (void) const;
  void SetPayloadId (uint32_t payloadId);
  void SetPayloadSize (uint32_t payloadSize);
  void SetIndex (uint32_t index);

  static TypeId GetTypeId (void);
  TypeId GetInstanceTypeId (void) const;
  void Print (std::ostream &os) const;
  uint32_t GetSerializedSize (void) const;
  void Serialize (Buffer::Iterator start) const;
  uint32_t Deserialize (Buffer::Iterator start);
private:
  uint32_t m_payloadId;
  uint32_t m_payloadSize;
  uint32_t m_index;
};
}
--------------------------------------------------
\end{lstlisting}
\end{footnotesize}


\subsection{CacheCast packet tag}



\section{Network support}



\section{Server support}
The CacheCast system relies on CacheCast support on the server in order to 
remove redundant payload in the network. This server support should send all 
packets sequentially onto the link and truncate all packets beside the first 
one. This removes redundant payload from the link and forms a packet train.

In the Linux operating system this functionality is provided through a system 
call (\texttt{msend}) and a kernel module. In the ns-3 network simulator there 
is no abstraction of an (operating) system. Hence, this server support mechanism must 
be handled in a different way than in the Linux operating system.

The CacheCast server support consists mainly of two parts; the interface to the 
applications and the underlying packet handling mechanisms. The implementation 
in ns-3 closely resembles this division. The API exposed to the applications is 
discussed in section \ref{api} and the underlying handling of 
packets and addition of the CacheCast header is discussed in the following section.

\subsection{Underlying packet handling mechanism}
The tasks of the underlying packet handling mechanism in CacheCast is to ensure 
that the packets are put onto the link in a tight chain and to add the CacheCast 
header to each packet. In the Linux implementation of CacheCast this mechanism 
is handled by a kernel module located between the network layer and the link 
layer. In the ns-3 network simulator the abstraction of the physical and the 
data link layer is modeled by a NetDevice. This NetDevice receives a 
packet from the network layer, adds link layer headers, and puts the packet onto 
the channel. Because of this design we have chosen to create a new NetDevice 
which supports CacheCast. This NetDevice is called CacheCastServerNetDevice and 
is derived from the PointToPointNetDevice. Since CacheCastServerNetDevice has 
PointToPointNetDevice as a base class we can take advantage of the already 
existing features in the PointToPointNetDevice. And since CacheCast is only supported on 
point-to-point links the CacheCastServerNetDevice need not be generalized into 
supporting arbitrary link level technologies.

In the Linux implementation the CacheCast packets is intercepted just before the 
link layer. In our implementation in ns-3 this interception will happen in the 
NetDevice. All NetDevices in ns-3 have a virtual function called Send(). This function is 
called from the network layer with the packet as an argument and this is where 
we will intercept the packet.


The \texttt{CacheCastServerNetDevice} is the 

CacheCastPid


\subsection{Application programming interface (API)\label{api}}

- first suggestion
- second suggestion
- what we ended up doing



\section{Helpers}


\section{Evaluation}


\section{Conclusion}
The conclusion goes here.



% \appendices
% \section{Proof of the First Zonklar Equation}
% Appendix one text goes here.
% 
% % you can choose not to have a title for an appendix
% % if you want by leaving the argument blank
% \section{}
% Appendix two text goes here.


\section*{Acknowledgment}
The authors would like to thank...
% references section

%\bibliographystyle{IEEEtran}
% argument is your BibTeX string definitions and bibliography database(s)
%\bibliography{IEEEabrv,../bib/paper}


%%---------------------------------------------------------------------------%%
\end{document}


% EXAMPLES

%\begin{figure}[!t]
%\centering
%\includegraphics[width=2.5in]{myfigure}
%\caption{Simulation Results}
%\label{fig_sim}
%\end{figure}

%\begin{table}[!t]
%% increase table row spacing, adjust to taste
%\renewcommand{\arraystretch}{1.3}
% if using array.sty, it might be a good idea to tweak the value of
% \extrarowheight as needed to properly center the text within the cells
%\caption{An Example of a Table}
%\label{table_example}
%\centering
%% Some packages, such as MDW tools, offer better commands for making tables
%% than the plain LaTeX2e tabular which is used here.
%\begin{tabular}{|c||c|}
%\hline
%One & Two\\
%\hline
%Three & Four\\
%\hline
%\end{tabular}
%\end{table}

